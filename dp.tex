\documentclass{bakalarka}[12pt]
\usepackage[utf8]{inputenc} 
\usepackage[czech]{babel}
\usepackage{ae}
\usepackage{fancyhdr}
\usepackage[pdftex]{graphicx}
\usepackage{enumerate}
\usepackage[thinlines]{easytable}
\usepackage{geometry}
\usepackage{caption}
\usepackage{amsmath,amsfonts,amssymb} 
\usepackage{graphicx}

\newgeometry{left=3.5cm,right=2.5cm,bottom=0cm,top=2.5cm}
%\documentclass[12pt]{extbook}
%\usepackage[a4paper,inner=3.5cm,pdftex]{geometry}




\author{Ladislav Hlom}	
\title{Automatická klasifikace vícejazyčných dokumentů}
\titlet{}
\titlett{}
\university{Západočeská univerzita v Plzni}
\faculty{Fakulta aplikovaných věd}
\department{Katedra informatiky a výpočetní techniky}
\subject{Diplomová práce}
\town{Plzeň}
\begin{document}
\pagenumbering{gobble}
\pagestyle{fancy}
\renewcommand{\chaptermark}[1]{\markboth{\textit{#1}}{}}
\renewcommand{\sectionmark}[1]{\markright{\textit{#1}}{}}
\cfoot{\thepage}
\lhead{\leftmark}
\rhead{\rightmark}
\maketitle

\chapter*{Prohlášení}
\thispagestyle{empty}
Prohlašuji, že jsem diplomovou práci vypracoval samostatně a výhradně s~použitím citovaných pramenů.
\vskip 1.5em
V Plzni dne \today
\vskip 0.7em
\hskip 9cm Ladislav Hlom
\chapter*{Abstract}
\thispagestyle{empty}
Cílem práce bylo ...
\tableofcontents
\pagestyle{fancy}
\renewcommand{\chaptermark}[1]{\markboth{\textit{#1}}{}}
\renewcommand{\sectionmark}[1]{\markright{\textit{#1}}{}}
\cfoot{\thepage}
\lhead{\leftmark}
\rhead{\rightmark}
\parskip 1em
 \pagenumbering{arabic} 
\chapter{Úvod}
1.Prostudujte vybrané metody pro automatický překlad.\\\\
2. Seznamte se se základními přístupy pro klasifikaci/kategorizaci
textových dokumentů. Zaměřte se na metody klasifikace do více tříd.\\\\
3. Analyzujte datové kolekce dodané od České tiskové kanceláře. Kolekce
doplňte o alespoň dvě další volně dostupné v jiných jazycích.\\\\
4. Na základě studie literatury navrhněte a implementujte dvě metody
automatické klasifikace vícejazyčných dokumentů.\\\\
5. Funkčnost metod otestujte na dostupných datech a zhodnoťte dosažené
výsledky.\\\\
6. Na základě dosažených výsledků navrhněte další rozšíření/vylepšení

knizka pro klasifikaci. http://nlp.stanford.edu/IR-book/html/htmledition/irbook.html

\chapter{Automatický překlad}


Rule Based approach:
 rule-based MT has two approaches: Interlingua and 
transfer. Rule-Based MT Systems rely on different levels of linguistic rules for 
translation. This MT research paradigm has been named rule-based MT due to 
the use of linguistic rules of diverse natures. For instance, rules are used for 
lexical transfer, morphology, syntactic an
alysis, syntactic generation, etc. In 
RBMT the translation process consists of: 
-
 Analyzing input text morphologically, syntactically and semantically. 
-
 Generating text via structural conversions based on internal structures. 
Proceedings ICWIT 2012                                                                                      165 
The steps mentioned above make use of a dictionary and a grammar, which must 
be developed by linguists. This requirement is the main problem of RBMT as it 
is a time-consuming process to collect a
nd spell out this knowledge, frequently 
referred as knowledge acquisition problem. It is not just very hard to develop 
and maintain the rules in this type of system, but one is not guaranteed to get the 
system to operate as well as before the addition of a new rule. RBMT systems 
are large-scale rule based systems; wher
eas their computational cost is high, 
since they must implement all aspects 
whether syntactic, semantic, structural 
transfer etc. as rules [14]. 
•
Corpus-based approach: 
Corpus-Based Machine Translation, also referred as 
data driven machine translation, is an alternative approach for machine 
translation to overcome the knowledge acquisition problem of rule-based 
machine translation. There are two 
types of CBMT Statistical Machine 
Translation (SMT) and Example-Based Machine Translation (EBMT). Corpus-
based MT automatically acquires the translation knowledge or models from 
bilingual corpora. Since this approach has been designed to work on large sizes 
of data, it has been named Corpus-Based MT ([17], [18], [16] and [15]). 
•
Hybride approach:
 Some recent work has focused 
on hybrid approaches that 
combine the transfer approa
ch with one of 
the corpus–based 
approaches. This 
was designed to work with fewer amounts of resources and depend on the 
learning and training of transfer rules.
 The main idea in this approach is to 
automatically learn syntactic transfer rules from limited amounts of word-
aligned data. This data contains all the needed information for parsing, transfer, 
and generation of the sentences ( [19] and [20]). The following section covers 
part of the MT literature that gives details of specific systems for deriving the 
appropriate translation us
ing different approaches


\chapter{Automatická klasifikace}


\chapter{Použitá metoda}
Dokumenty mohou být v rozdílném jazyce. Je tedy potřeba dokument nejprve přeložit a poté ho lze klasifikovat. Jako výchozí jazyk pro klasifikaci byla zvolena čeština. Nejprve tedy bude zjištěn výchozí jazyk dokumentu a v případě, že nebude v českém jazyce, pak do něho bude přeložen. Poté bude možné dokument klasifikovat. Klasifikátory musí být nejdříve natrénovány na trénovacích dokumentech a poté již mohou dokumenty klasifikovat do jedné z natrénovaných kategorií.

\section{Přeložení dokumentu} 
Dokumenty se mohou vyskytovat v různých jazycích. Očekávané jsou dokumenty v angličtině a němčině, které je nutné přeložit do češtiny. Pro přeložení budou použity alespoň dva různé nástroje. Bude ohodnocena kvalita překladu jednotlivých nástrojů a bude vyhodnoceno jak ovlivňuje kvalita překladu následnou klasifikaci. Pro překlad byly vybrány dva rozdílné nástroje - Moses, Google Translate.

\subsection{Moses} 
Moses je implementací statistického řešení strojového překladu. Jedná se o open-source projekt pod licencí LGPL. Mezi jeho výhody patří možnost bezplatného využití.

\subsection{Google Translate}
Google Translate je implementací statistického řešení strojového překladu.

\subsection{Ohodnocení kvality překladu}
V rámci práce bude zjištěn vliv kvality překladu na přesnost klasifikace. Pro tyto účely existuje několik možných metrik. Pro stanovení metriky je potřeba mít text k překladu a přeložený text, nejlépe s několika možnými variantami překladu. Pro stanovení metriky bude v práci použit česko-anglický korpus, který sice poskytuje pouze 1 variantu překladu. Ale pro získání informace o kvalitě překladu je to dostačující.

\subsubsection{Bleu}
http://homepages.inf.ed.ac.uk/pkoehn/publications/esslli-slides-day4.pdf

\section{Klasifikace} 
Pro klasifikaci byla zvolena možnost trénování s učitelem. Z klasifikačních metod byly zvoleny následující - Bayes, Maximální entropie, SVM.

\subsection{Dokumenty ke klasifikaci}
Data pro trénování a testování klasifikátorů poskytla Česká tisková kancelář (dále jen ČTK). Dodaná data se nachází v xml formátu a dále jsou k dispozici textové soubory. Textová data obsahují správné kategorie v názvu souboru a uvnitř souborů se nachází tisková zpráva. V xml formátu obsahují data mnoho atributů. Použité atributy jsou kategorie, která určuje do které kategorie dokument patří. Dalším polem je agentura, podle které lze zjistit v jakém jazyku je dokument. 
Dalším je samotná tisková zpráva.

\subsection{Nástroje ke klasifikaci}
Jako programovací jazyk pro otestování řešení byla zvolena Java. Pro klasifikaci byla na základě doporučení zvolena knihovna Brainy.
\subsubsection{Brainy}
Jedná se o knihovnu v Javě pro klasifikaci, která pochází z naší fakulty. ...

\subsection{Ohodnocení přesnosti klasifikace}
http://dl.acm.org/citation.cfm?id=944974
Pro ohodnocení přesnosti je použita metoda  F-measure. Ta se vypočte s použitím recall a precission. Při výpočtu musí být nejprve určeny následující hodnoty: \\\\
TP (true positive) - počet správně zařazených dokumentů \\
TN (true negative) - počet nesprávně zařazených dokumentů \\
FN (false negative) - počet nesprávně nezařazených dokumentů \\\\
Precission je zlomek výsledků, které jsou relevantní. \\ \\
$precission = \dfrac{tp}{tp + fp}$   \\ \\
Recall je zlomek výsledků, které jsou získány ze všech požadovaných.  \\ \\
$recall = \dfrac{tp}{tp + fn}$   \\ \\
Tradiční  F-measure je harmonický průměr precission a recall. \\ \\
$F-measure = \dfrac{2 \cdot( precission \cdot recall )}{precission + recall}$

    
\subsection{Křížová validace}   
Vstupní množina dat je rozdělena na podmnožiny. Jedna podmnožina slouží jako testovací množina, zbylé podmnožiny slouží jako trénovací množiny. Klasifikátor natrénuje model na trénovací množině a pomocí testovací množiny testuje přesnost a výkonnost tohoto modelu. Tento proces se několikrát opakuje, pokaždé s jinou podmnožinou tvořící trénovací a testovací množinu.

\subsection{Metody klasifikace}
Pro klasifikaci byly zvoleny následující varianty.

\subsubsection{Class \& Complement}

\subsubsection{Threshold}

\section{Testovací program} 
Program vytvořený pro otestování navrženého řešení. Cílem je umožnit otestovat řešení na xml datech a na dodaných textových dokumentech. Dále se nachází diagram programu viz obr ~\ref{overflow} spolu s popisem jednotlivých metod.
\begin{figure}[ht!]
\centering
\includegraphics[width=140mm]{img/diagram_program.png}
\caption{Diagram testovacího programu \label{overflow}}
\end{figure}

\noindent LoadAllDocuments - Metoda načte všechny soubory a uloží je do struktury pro další zpracování. 

\noindent LoadXmlData - načte všechna XML data od ČTK a potřebné údaje si uloží spolu s jednotlivými zprávami. Poté rozdělí jednotlivé dokumenty podle jazyku. Pro jazyky které je potřeba přeložit připraví textový soubor k přeložení. Tento soubor je přeložen do češtiny externím programem.

\noindent LoadTranslatedDocuments -  přeložené zprávy se nahrají ze souboru a aktualizují se záznamy jednotlivých dokumentů.

\noindent CategoriesPreparation - z načtených dokumentů se získají všechny kategorie.

\noindent RemoveCategoriesWithLowOccurences - odstraní se kategorie, které obsahují menší počet trénovacích dokumentů než je stanovená mez. Tyto kategorie jsou odstraněny také z dokumentů a pokud dokument již nepatří do žádné další kategorie, pak je vyhozen.

\noindent CreateTrainingData - připraví data pro trénování. Odstraní ze zpráv html znaky a interpunkci.

\noindent TestThreshold - natrénuje a otestuje klasifikátor metodou Threshold.

\noindent TestClass - natrénuje a otestuje klasifikátor metodou Class \& Complement

\noindent WriteResults - zapíše výsledky metriky do souboru.

\appendix
\bibliographystyle{csplainnat}
\bibliography{bakalarka}
\end{document}
